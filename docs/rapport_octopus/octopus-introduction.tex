\section*{Introduction}
\addcontentsline{toc}{section}{Introduction}
\paragraph{}
Dans le cadre de la deuxième année d'étude du cycle ingénieur dans la formation Sécurité et Technologies Informatique de l'INSA Centre Val de Loire, nous travaillons sur un projet en binome.
Le sujet est "Supervision et audit de la sécurité système dans un réseau". Les professeurs tuteurs sur ce projet sont M. Briffaut et M. Szpieg.
\paragraph{}
L'objectif est donc de réaliser un logiciel maitre/esclave avec des fonctionnalités de supervision des esclaves et de gestion à distance de ceux-ci. 
La supervision et la gestion attendue concernent la sécurité des systèmes esclaves afin de pouvoir certifier la fiabilité des clients, la confidentialité des données, la tracabilité des communications et un système informatique sain pour l'entreprise.
\paragraph{}
Le cahier des charges demande donc un outil d'administration qui sera utilisé par des administrateurs réseaux ou des responsables sécurité d'entreprise. 
C'est un outil qui sera facilement déployable sur un parc informatique important et diversifié. Les fonctionnalités d'administration et de supervision des machines clientes seront avancées et pourront se faire de manière pratique via une interface Web.
\paragraph{}
Ce rapport vous présente de manière vulgarisée le fonctionnement de notre logiciel. Nous avons choisi de présenter le fonctionnement général du programme dans un premier temps. 
Ensuite, nous expliciterons les algorithmes de fonctionnement du serveur puis du client. Et enfin, nous présentons un peu plus en détail les points de sécurité de notre logiciel qui sont importants.
\paragraph{}
Nous proposons donc une solution légère mais offrant de nombreuses fonctionnalités permettant la supervision et l'audit de la sécurité des systèmes dans un réseau. 
Nous vous proposons une solution fiable, sécurisée et facile d'accès. Vous pourrez la découvrir dans ce rapport.